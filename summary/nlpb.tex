\section{MIMO Controller for Power Balancing}

In recent years, hardware has been taking more direct control of processor voltage and frequency, making fine-tuned software-management of DVFS obsolete.
Fine-grained \emph{power capping} is now the preferred mechanism for managing performance and power tradeoffs, evidenced by the introduction of power capping implementations like Intel Running Average Power Limit (RAPL) \cite{RAPL} and the move from Speed Step to Speed Shift.

Nodes in HPC clusters suffer \emph{non-uniformity} in performance and power consumption due to both manufacturing process variation and imbalance in application workloads.
We propose a \interface{Multiple Input, Multiple Output (MIMO) Proportional Integral Derivative (PID) controller} to shift power allocations between nodes to eliminate tail idle times in job iterations, thus maximizing application throughput while respecting a global power cap.
Unlike heuristic approaches to power management, control theory provides \emph{formal guarantees} of \emph{convergence} as well as \emph{robustness} to application, system, and measurement noise \cite{Hellerstein2004a}.
These guarantees hold even across application \emph{phases}.
\figref{runtime-nlpb} demonstrates the control-theoretical design.

\begin{figure}[t]
  \tikzset{%
  app/.style    = {draw, thin, rectangle, minimum height = 2em,
    minimum width = 2em, fill=black!25},
  block/.style    = {draw, thick, rectangle, minimum height = 2.5em,
    minimum width = 2.5em},
  blockres/.style    = {draw, thick, rectangle, minimum height = 2.5em,
    minimum width = 2.5em, fill=green!25},
  biblock/.style  = {draw, thick, rectangle, minimum height = 5.5em,
    minimum width = 6em, fill=red!25},
  sum/.style      = {draw, circle, node distance = 2cm}, % Adder
  input/.style    = {coordinate}, % Input
  output/.style   = {coordinate} % Output
}

\tikzstyle{vecArrow} = [thick, decoration={markings,mark=at position
   1 with {\arrow[semithick]{open triangle 60}}},
   double distance=1.4pt, shorten >= 5.5pt,
   preaction = {decorate},
   postaction = {draw,line width=1.4pt, white,shorten >= 4.5pt}]
\tikzstyle{vecLine} = [thick, double distance=1.4pt, shorten >= 5.5pt]
\tikzstyle{innerWhite} = [semithick, white,line width=1.4pt, shorten >= 4.5pt]

\begin{tikzpicture}[scale=1.0,transform shape, auto, thick, node distance=1.5cm, >=triangle 45]

\draw
  % Drawing the top blocks
  node [input, name=goalaccuracy] {} 
  node [left of=goalaccuracy, node distance=0.35mm]{}
  node [sum, right of=goalaccuracy] (sumaccuracy) {} % negative feedback
  node [block, right of=sumaccuracy, align=center, node distance=2cm] (controlaccuracy) 
    {~Controller~}
  % node [block, right of=controlaccuracy, align=center, node distance=3.0cm] (translateaccuracy) 
  %   {Mapper}
  % node [blockres, above of=controlaccuracy, align=center, node distance=5cm] (resourcefile) 
  %   {Initial\\Power Caps}
;
  % Connectng lines
\draw[vecArrow](goalaccuracy) -- node[align=center] {Reference\\Idle Times\\($\vec{0}$)}(sumaccuracy);
\draw[vecArrow](sumaccuracy) -- node[align=center] {Relative\\Timing\\Error}(controlaccuracy);
% \draw[->](controlaccuracy) -- node[align=center] {Generic\\Control\\Signal}(translateaccuracy);
% \draw[vecArrow](resourcefile) -- (controlaccuracy);

% Draw software system
\draw
  node [biblock, right of=controlaccuracy, node distance=3cm, align=center, yshift=-0.2cm] (system)
    {\\System\\\\\\}
;
\draw
  node [app, right of=controlaccuracy, node distance=3cm, align=center, yshift=-0.5cm] (software)
    {Distributed\\Application}
;

% lines from translators to software
\draw[vecArrow](controlaccuracy.east) -- node [name=ka,align=center]{Power\\Caps} (controlaccuracy.east -| system.west);

% Connectng lines
\coordinate (feedbackup) at ([yshift=-0.6cm]sumaccuracy.south);
\draw[vecLine](software.west |- feedbackup) -| node [near end,align=center] {\\Measured\\Idle Times} (feedbackup);
\draw[vecArrow](feedbackup) -- node[pos=0.99] {$-$} (sumaccuracy);

\draw[innerWhite](software.west |- feedbackup) -| node [near end,align=center] {} (feedbackup);
% \draw[innerWhite](feedbackup) -- node[pos=0.99] {} (sumaccuracy);

\end{tikzpicture}
  \caption{MIMO PID controller closed loop feedback control.}
  \label{fig:runtime-nlpb}
\end{figure}

Given a cluster with $n$ nodes and global power cap $\Gamma$, the PID controller computes a new power signal vector $\vec{u}$ of size $n$ in each iteration $t$:
\begin{eqnarray}
\vec{u}(t) = \vec{u}(t-1) + K_I \cdot \vec{e}(t)
\end{eqnarray}
$K_I$ is the ratio of control change and $\vec{e}(t)$ are the node idle times, divided by their mean and normalized around $0$.
This formulation ensures that the entire global power cap is re-allocated in each iteration.
Formally:
\begin{eqnarray}
\sum_{i=1}^{n} e_i(t) = 0 \implies \sum_{i=1}^{n} u_i(t) = \sum_{i=1}^{n} u_i(t-1) = \Gamma
\end{eqnarray}
% $K_I$ is currently a fixed scalar value decided at initialization time, but we are exploring ways to predict it dynamically at runtime, \eg using a Kalman filter.

\figref{nlpb-sim} simulates meeting a global power cap of 180 Watts on a theoretically imbalanced 4-node cluster ($\Gamma=180$, $n=4$) using a MPI+OpenMP application.
It begins with evenly distributed power caps, \ie $u_i(0) = \frac{\Gamma}{n}$ for each node $i$.
The controller quickly re-balances power caps so that nodes finish their jobs with near-zero tail idle times, thus improving the total application performance.

\begin{figure}[t]
  \begin{tikzpicture}
\begin{centering}

\definecolor{s1}{RGB}{228, 26, 28}
\definecolor{s2}{RGB}{55, 126, 184}
\definecolor{s3}{RGB}{77, 175, 74}
\definecolor{s4}{RGB}{152, 78, 163}
\definecolor{s5}{RGB}{255, 127, 0}

\begin{groupplot}[
    group style={
        group name=plots,
        group size=1 by 2,
        xlabels at=edge bottom,
        xticklabels at=edge bottom,
        vertical sep=10pt
    },
height=8cm,
width=0.8\columnwidth,
xmajorgrids,
ymajorgrids,
grid style={dashed},
xmin=0,
xmax=10,
yticklabel pos=left,
enlargelimits=false,
tick align = outside,
tick style={white},
xticklabel shift={-5pt},
yticklabel shift={-5pt},
ylabel shift={-2pt},
ylabel style={align=center},
unbounded coords=jump,
legend cell align=left, 
legend style={ column sep=1ex },
]


\nextgroupplot[ylabel={Idle Time (S)}, % Performance
% xlabel={$time$ [iteration]},
% xlabel near ticks,
xtick={0,1,2,3,4,5,6,7,8,9,10},
ytick={0,2,4,6,8,10,12,14},
yticklabels={0,2,4,6,8,10,12,~~~14},
% yticklabel style={font=\footnotesize},
ymin=0,
ymax=14,
legend entries={{$\mathsf{Process~0}$},{$\mathsf{Process~1}$},{$\mathsf{Process~2}$},{$\mathsf{Process~3}$}},
legend style={draw=none,at={(0.5,1.25)},anchor=north,legend columns=4,line width=5pt},
]
\addplot[ultra thick, solid, color=s1] table[x index=0,y index=10,col sep=space] {img/nlpb.log};
\addplot[ultra thick, solid, color=s2] table[x index=0,y index=13,col sep=space] {img/nlpb.log};
\addplot[ultra thick, solid, color=s3] table[x index=0,y index=16,col sep=space] {img/nlpb.log};
\addplot[ultra thick, solid, color=s4] table[x index=0,y index=19,col sep=space] {img/nlpb.log};
\addplot[thick, solid, black] coordinates {(0, 45) (10, 45)};
% \addplot[ultra thick, dashed, black] coordinates {(1500,0) (1500, 2)};
% \addplot[ultra thick, dashed, black] coordinates {(3000,0) (3000, 2)};

\nextgroupplot[ylabel={Power Cap (W)}, % Performance
xlabel={$time$ [iteration]},
xlabel near ticks,
xtick={0,1,2,3,4,5,6,7,8,9,10},
ytick={44,44.25,44.5,44.75,45,45.25,45.5,45.75,46,46.25,46.5},
yticklabels={44.0,,44.5,,45.0,,45.5,,46.0,,},
% yticklabel style={font=\footnotesize},
ymin=44,
ymax=46.5,
% legend entries={{$\mathsf{Process~0}$},{$\mathsf{Process~1}$},{$\mathsf{Process~2}$},{$\mathsf{Process~3}$}},
% legend style={draw=none,at={(0.5,1.2)},anchor=north,legend columns=4,line width=5pt},
]
\addplot[ultra thick, solid, color=s1] table[x index=0,y index=12,col sep=space] {img/nlpb.log};
\addplot[ultra thick, solid, color=s2] table[x index=0,y index=15,col sep=space] {img/nlpb.log};
\addplot[ultra thick, solid, color=s3] table[x index=0,y index=18,col sep=space] {img/nlpb.log};
\addplot[ultra thick, solid, color=s4] table[x index=0,y index=21,col sep=space] {img/nlpb.log};
\addplot[thick, solid, black] coordinates {(0, 45) (10, 45)};
% \addplot[ultra thick, dashed, black] coordinates {(1500,0) (1500, 2)};
% \addplot[ultra thick, dashed, black] coordinates {(3000,0) (3000, 2)};


\end{groupplot}
\end{centering}

\end{tikzpicture}

  \caption{Adapting node power caps between iterations.}
  \label{fig:nlpb-sim}
  \vskip -1.8em
\end{figure}
