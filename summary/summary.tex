% \documentclass[9pt,nocopyrightspace]{sig-alternate}
% \documentclass[conference]{sig-alternate-05-2015}
\documentclass[sigconf,natbib=false]{acmart}

% =============================================================================

% \usepackage[normalem]{ulem}
% \usepackage{float}
% \usepackage{url}
% \usepackage{makeidx}
% \usepackage{graphicx}
% \usepackage{graphics}
% \usepackage{multicol}
% \usepackage{xspace}
% \usepackage{chngpage}
%\usepackage{narrow}
%\usepackage{fullpage}
%\usepackage{subfigure}
% \usepackage{subfig}
% \usepackage{multirow}
% \usepackage{rotating}
% \usepackage{color}
%\usepackage{algorithmic2e}
% \usepackage{algpseudocode}
% \usepackage{algorithm}
% \usepackage{savesym}
% \usepackage{amsmath}
% \usepackage{amssymb,latexsym}
% \everymath{\displaystyle}
% \savesymbol{iint}
% \usepackage{txfonts}
% \restoresymbol{TXF}{iint}
%\usepackage{mathptmx} % amsmath makes the paper longer
% \usepackage{mathrsfs}
% \usepackage{leading}
%\usepackage{algorithmic}
%\usepackage{algorithm}
%\usepackage{fullpage}
%\usepackage[font={small,bf},nooneline]{caption}
% \usepackage{hyphenat}
% \usepackage{leading}
% \let\labelindent\relax
% \usepackage{enumitem}
% \usepackage[natbib=true,backend=bibtex,firstinits=true,style=numeric-comp,sorting=nyt,defernumbers,maxnames=2,maxcitenames=2,doi=false,isbn=false,url=false]{biblatex}
\usepackage[natbib=true,backend=bibtex,firstinits=true,style=numeric-comp,sorting=nyt,defernumbers,maxnames=99,maxcitenames=99,doi=false,isbn=false,url=false]{biblatex}
% \usepackage{balance}


\definecolor{mygreen}{rgb}{0,0.6,0}
\definecolor{mygray}{rgb}{0.5,0.5,0.5}
\definecolor{mymauve}{rgb}{0.58,0,0.82}
\usepackage{listings}
\lstset{ %
  backgroundcolor=\color{white},   % choose the background color; you must add \usepackage{color} or \usepackage{xcolor}
  basicstyle=\scriptsize\ttfamily, % the size of the fonts that are used for the code
  breakatwhitespace=false,         % sets if automatic breaks should only happen at whitespace
  breaklines=true,                 % sets automatic line breaking
  captionpos=b,                    % sets the caption-position to bottom
  commentstyle=\color{mygreen},    % comment style
  deletekeywords={...},            % if you want to delete keywords from the given language
  escapeinside={\%*}{*)},          % if you want to add LaTeX within your code
  extendedchars=true,              % lets you use non-ASCII characters; for 8-bits encodings only, does not work with UTF-8
  frame=leftline,                  % adds a frame around the code
  keepspaces=true,                 % keeps spaces in text, useful for keeping indentation of code (possibly needs columns=flexible)
  keywordstyle=\color{blue},       % keyword style
  morekeywords={*,...},            % if you want to add more keywords to the set
  numbers=left,                    % where to put the line-numbers; possible values are (none, left, right)
  numbersep=5pt,                   % how far the line-numbers are from the code
  numberstyle=\tiny\color{mygray}, % the style that is used for the line-numbers
  rulecolor=\color{black},         % if not set, the frame-color may be changed on line-breaks within not-black text (e.g. comments (green here))
  showspaces=false,                % show spaces everywhere adding particular underscores; it overrides 'showstringspaces'
  showstringspaces=false,          % underline spaces within strings only
  showtabs=false,                  % show tabs within strings adding particular underscores
  stepnumber=1,                    % the step between two line-numbers. If it's 1, each line will be numbered
  stringstyle=\color{black},     % string literal style
  tabsize=1,                       % sets default tabsize to 2 spaces
  title=\lstname                   % show the filename of files included with \lstinputlisting; also try caption instead of title
}

\usepackage{pgfplots}
% options for pgfplots
\pgfplotsset{compat=1.8,compat/show suggested version=false}
\usetikzlibrary{calc,trees,arrows,patterns,plotmarks,shapes,snakes,er,3d,automata,backgrounds,topaths,decorations.pathmorphing,decorations.markings}
%\pgfplotsset{compat=newest}
\pgfplotsset{
   /pgfplots/bar  cycle  list/.style={/pgfplots/cycle  list={%
        {black,fill=black!30!white,mark=none},%
        {black,fill=red!30!white,mark=none},%
        {black,fill=green!30!white,mark=none},%
        {black,fill=yellow!30!white,mark=none},%
        {black,fill=brown!30!white,mark=none},%
     }
   },
}
% begin of externalization
\usetikzlibrary{external}
\tikzexternalize[prefix=out/]
\tikzexternalize
% don't externalize todonotes
%\makeatletter
%\renewcommand{\todo}[2][]{\tikzexternaldisable\@todo[#1]{#2}\tikzexternalenable}
%\makeatother
% end of externalization
\usetikzlibrary{patterns}
\usepgfplotslibrary{groupplots}
\pgfplotsset{
every axis label/.append style={font=\footnotesize},
tick label style={font=\footnotesize},
}
%\usepackage[normalem]{ulem}
%\setlength{\abovecaptionskip}{2pt plus 2pt minus 2pt}
%\setlist{noitemsep,topsep=0pt}

%\setlength{\itemsep}{0pt}
%\setlength{\topsep}{0pt}
%\setlength{\partopsep}{0pt}
%\setlength{\parsep}{1pt}
%\setlength{\parskip}{2pt}
%\setlength{\abovecaptionskip}{2pt plus 4pt minus 0pt}
%\setlength{\textfloatsep}{1pt}

\bibliography{../seec}
\renewcommand{\bibfont}{\footnotesize}
%\setlength\bibitemsep{0pt}

%reduce space around equations
% \makeatletter
% \g@addto@macro\normalsize{%
%   \setlength\abovedisplayskip{4pt plus 2pt minus 1pt}
%   \setlength\belowdisplayskip{4pt plus 2pt minus 1pt}
%   \setlength\abovedisplayshortskip{4pt plus 2pt minus 1pt}
%   \setlength\belowdisplayshortskip{4pt plus 2pt minus 1pt}
% }
% \makeatother

% \conferenceinfo{SC '17}{November 12--17, 2017, Denver, CO, USA}
% \doi{01.0001/0000001.0000001}
% \setcopyright{acmcopyright}

% \newif{\ifanonymous}
% \anonymoustrue

%\newcommand{\comment}[1]{}
\newcommand{\cutout}[1]{}
\newcommand{\smallcaption}[1]{\caption[#1]{{\protect\small \protect\bf #1}}}
\newcommand{\dids}{{\sc dids}}

%\graphicspath{{figs/}}

% some useful shortcuts
\newcommand{\ie}{\textit{i.e., }}
\newcommand{\eg}{\textit{e.g., }}
\newcommand{\CC}{C\nolinebreak\hspace{-.05em}\raisebox{.5ex}{\tiny\bf +}\nolinebreak\hspace{-.10em}\raisebox{.5ex}{\tiny\bf +}}

% units for results
\newcommand{\us}{\,$\mu$s}
\newcommand{\ms}{\,ms}
\newcommand{\KB}{\,KB}
\newcommand{\MB}{\,MB}
\newcommand{\GB}{\,GB}
\newcommand{\MHz}{\,MHz}
\newcommand{\GHz}{\,GHz}

%\newcommand{\SYSTEM}{POET}
% \newcommand{\system}{poet}
% new latex commands:
%   Remove long section
\newcommand{\PUNT}[1]{}
%   Label work to be done
\definecolor{gray}{gray}{0.75}
\newcommand{\TODO}[1]{\textcolor{gray}{\textbf{\ [TODO:\ #1]\ }}}
%\newcommand{\TODO}[1]{}
\newcommand{\FIX}[1] {\textcolor{red}{\textbf{\ [FIX:\ #1]\ }}}
%   Referencing various pieces of the document:
\newcommand{\figref}[1]{Figure~\ref{fig:#1}}
\newcommand{\figsref}[2]{Figures~\ref{fig:#1} and~\ref{fig:#2}}
\newcommand{\figrref}[2]{Figures~\ref{fig:#1}--\ref{fig:#2}}
\newcommand{\secref}[1]{Section~\ref{sec:#1}}
\newcommand{\secsref}[2]{Sections~\ref{sec:#1} and~\ref{sec:#2}}
\newcommand{\eqnref}[1]{Eqn.~\ref{eqn:#1}}
\newcommand{\eqnsref}[2]{Eqns.~\ref{eqn:#1} and~\ref{eqn:#2}}
\newcommand{\eqnrref}[2]{Eqns.~\ref{eqn:#1}--\ref{eqn:#2}}
\newcommand{\insref}[1]{Instruction~\ref{ins:#1}}
\newcommand{\tblref}[1]{Table~\ref{tbl:#1}}
\newcommand{\tblsref}[2]{Tables~\ref{tbl:#1} and~\ref{tbl:#2}}
\newcommand{\appref}[1]{Appendix~\ref{app:#1}}
\newcommand{\algoref}[1]{Algorithm~\ref{algo:#1}}

% Custom hyphenation rules
\hyphenation{Ang-strom}

%\DeclareMathOperator{\minimize}{minimize}
%\DeclareMathOperator{\st}{s.t.}
%\DeclareMathOperator*{\argmin}{argmin}
%\DeclareMathOperator*{\argmax}{argmax}
\newcommand{\argmin}{\arg\!\min}
\newcommand{\argmax}{\arg\!\max}
\newcommand{\minimize}{minimize}
\newcommand{\maximize}{maximize}
\newcommand{\optimize}{optimize}
\newcommand{\st}{s.t.}

\newcommand{\app}[1]{\mbox{\texttt{#1}}}
\newcommand{\interface}[1]{\textbf{#1}}
\newcommand{\function}[1]{\mbox{\texttt{#1}}}
\newcommand{\struct}[1]{\emph{#1}}
\newcommand{\variable}[1]{\emph{#1}}

% Use bold for vectors instead of arrows
\renewcommand{\vec}[1]{\mathbf{#1}}

% =============================================================================

% Copyright
%\setcopyright{none}
\setcopyright{acmcopyright}
%\setcopyright{acmlicensed}
% \setcopyright{rightsretained}
%\setcopyright{usgov}
%\setcopyright{usgovmixed}
%\setcopyright{cagov}
%\setcopyright{cagovmixed}

% DOI
\acmDOI{10.475/123_4}

% ISBN
\acmISBN{123-4567-24-567/08/06}

%Conference
\acmConference[SC '17]{The International Conference for High Performance Computing, Networking, Storage and Analysis}{November 12--17, 2017}{Denver, CO, USA}
\acmYear{2017}
\copyrightyear{2017}


\acmArticle{4}
\acmPrice{15.00}

% =============================================================================

\begin{document}

% \date{}

\title{Energy Efficiency in HPC with Machine Learning and Control Theory}

\author{Connor Imes}
\authornote{Part of this research was conducted while Connor Imes was a Computing Sciences Summer Student at Lawrence Berkeley National Laboratory.}
\affiliation{%
  \institution{University of Chicago}
}
\email{ckimes@cs.uchicago.edu}

\author{Steven Hofmeyr}
\affiliation{%
  \institution{Lawrence Berkeley National Laboratory}
}
\email{shofmeyr@lbl.gov}

\author{Henry Hoffmann}
\affiliation{%
  \institution{University of Chicago}
}
\email{hankhoffmann@cs.uchicago.edu}

% The default list of authors is too long for headers}
% \renewcommand{\shortauthors}{C. Imes et al.}

% =============================================================================

\begin{abstract}
Performance and power management in High Performance Computing (HPC) has historically favored a race-to-idle approach in order to complete applications as quickly as possible, but this is not energy-efficient on modern systems.
As we move toward exascale and hardware over-provisioning, power management is becoming more critical than ever for HPC system administrators, opening the door for more balanced approaches to performance and power management.
We propose two projects to address balancing application performance and system power consumption in HPC \emph{during application runtime}, using closed loop feedback designs based on the Self-aware Computing Model to \emph{observe}, \emph{decide}, and \emph{act}.
\end{abstract}

%
% The code below should be generated by the tool at
% http://dl.acm.org/ccs.cfm
% Please copy and paste the code instead of the example below.
%
\begin{CCSXML}
<ccs2012>
  <concept>
    <concept_id>10002944.10011123.10011674</concept_id>
    <concept_desc>General and reference~Performance</concept_desc>
    <concept_significance>500</concept_significance>
  </concept>
  <concept>
    <concept_id>10002944.10011123.10011124</concept_id>
    <concept_desc>General and reference~Metrics</concept_desc>
    <concept_significance>500</concept_significance>
  </concept>
</ccs2012>
\end{CCSXML}
\ccsdesc[500]{General and reference~Performance}
\ccsdesc[500]{General and reference~Metrics}
\keywords{Energy Efficiency, Power Management, Runtime Control, Machine Learning, Control Theory}

\maketitle

% \begin{figure}[t]
%   \includegraphics[width=0.8\columnwidth]{../figures/seec.png}
%   \caption{The Self-aware Computing Model.}
%   \label{fig:runtime-ee}
% \end{figure}

\section{Machine Learning Classification for Energy Efficiency}

Tuning \textbf{core allocation (taskset)} and \textbf{dynamic voltage and frequency scaling (DVFS)} can save power and energy by power gating unneeded cores and scaling back active processors, \eg during I/O or for code not on the critical path.
Both are available in HPC job schedulers, though unfortunately today they are typically only configurable when a job is launched.

Other works have successfully used performance counters to manage performance and power consumption \cite{WuHPCComputer,Chetsa,Libutti2014}, including some that use statistical and machine learning approaches \cite{Sasaki,ShyeIDVFS,Alvarado,Curtis-Maury2008,LiIPDPS2010}.
However, these works mainly use estimation techniques that require significant modeling and computation.
By treating the goal of maximizing energy efficiency as a classification problem, we can avoid this pitfall.
Additionally, maximizing energy efficiency provides an optimal balance of performance and energy consumption while reducing dynamic power so it can be reallocated elsewhere.

\begin{figure}[t]
  \tikzset{%
  app/.style    = {draw, thin, rectangle, minimum height = 2.5em,
    minimum width = 5em, fill=black!25},
  block/.style    = {draw, thick, rectangle, minimum height = 3em,
    minimum width = 5em},
  blockres/.style    = {draw, thick, rectangle, minimum height = 3em,
    minimum width = 5em, fill=green!25},
  biblock/.style  = {draw, thick, rectangle, minimum height = 6em,
    minimum width = 6em, fill=red!25},
  sum/.style      = {draw, circle, node distance = 4cm}, % Adder
  input/.style    = {coordinate}, % Input
  output/.style   = {coordinate} % Output
}

\begin{tikzpicture}[scale=1.0,transform shape, auto, thick, node distance=1.5cm, >=triangle 45]

\draw
  % Drawing the top blocks
  % node [input, name=goalaccuracy] {} 
  % node [left of=goalaccuracy, node distance=0.35mm]{}
  % node [sum, right of=goalaccuracy] (sumaccuracy) {} % negative feedback
  node [block, align=center] (featureselection) 
    {Feature\\Selection}
  node [block, right of=featureselection, align=center, node distance=12cm] (classifier) 
    {Classifier}
  node [blockres, above of=classifier, align=center, node distance=5cm] (trainingdata) 
    {Training\\Data}
;
  % Connectng lines
% \draw[->](goalaccuracy) -- node[align=center] {Timing\\Goal}(sumaccuracy);
% \draw[->](sumaccuracy) -- node[align=center] {Timing\\Error}(featureselection);
\draw[->](featureselection) -- node[align=center] {Processed\\Data}(classifier);
\draw[->](trainingdata) -- (classifier);

% Draw software system
\draw
  node [biblock, right of=classifier, node distance=12cm, align=center, yshift=-1cm] (system)
    {\\System\\\\\\}
;
\draw
  node [app, right of=classifier, node distance=12cm, align=center, yshift=-2cm] (software)
    {Application}
;

% lines from translators to software
\draw[->](classifier.east) -- node [name=ka,align=center]{System\\Settings} (classifier.east -| system.west);

% Connectng lines
\coordinate (feedbackup) at ([yshift=-2cm]featureselection.south);
\draw (system.west |- feedbackup) -| node [near end,align=center] {PCM\\Sample} (feedbackup);
\draw[->](feedbackup) -- node[pos=0.99] {} (featureselection);

\end{tikzpicture}
  \caption{Machine Learning Classifier feedback control design.}
  \label{fig:runtime-ee}
\end{figure}

We propose using \interface{Machine Learning classifiers} to predict the most energy-efficient combination of taskset and DVFS settings \emph{during runtime} to maximize the work-to-energy ratio of an application running on a single node.
\figref{runtime-ee} demonstrates the feedback design.
Low-level \emph{hardware counter metrics}, available through tools like \app{PAPI} and \app{PCM}, train and drive our classifiers.
At regular time intervals, a classifier predicts the most energy-efficient setting to use based on measured application behavior, then applies the setting to the system.
As applications move through \emph{phases}, the predictions change accordingly.

We evaluate our approach on a quad-socket (160-logical-core) system with Intel Xeon E7-8870 processors and 512 GB of DRAM using HPC bioinformatic applications, which are often run on such large systems.
Naturally, it is important to demonstrate results on a single node before scaling to multiple nodes.
Training data is collected from characterizations of smaller HPC applications like the NAS Parallel Benchmarks, CoMD, HPGMG-FV, LULESH, and STREAM.
Principal Component Analysis determines which features are most relevant to energy efficiency---DRAM power, instructions per nominal CPU cycle (EXEC), L2/L3 cache hit/miss rates, and relative frequency excluding sleep time (AFREQ) are among the most useful.
We evaluate 6 common classifiers -- Gradient Boosting (GB), K-Nearest Neighbors (KNN), Random Forest (RF), Stochastic Gradient Descent (SGD), Support Vector Machine (SVM), and a Linear SVM.

\begin{figure}[t]
  \begin{tikzpicture}
\begin{centering}

\definecolor{s1}{RGB}{228, 26, 28}
\definecolor{s2}{RGB}{55, 126, 184}
\definecolor{s3}{RGB}{77, 175, 74}
\definecolor{s4}{RGB}{152, 78, 163}
\definecolor{s5}{RGB}{255, 127, 0}

\begin{groupplot}[
    group style={
        group name=plots,
        group size=1 by 1,
        xlabels at=edge bottom,
        xticklabels at=edge bottom,
        vertical sep=10pt
    },
height=3cm,
width=0.95\columnwidth,
xmajorgrids,
ymajorgrids,
grid style={dashed},
xmin=0,
xmax=4010,
yticklabel pos=left,
enlargelimits=false,
tick align = outside,
tick style={white},
xticklabel shift={-5pt},
yticklabel shift={-5pt},
ylabel shift={-2pt},
ylabel style={align=center, font=\scriptsize},
xlabel style={font=\scriptsize},
unbounded coords=jump,
xticklabel style={font=\scriptsize},
yticklabel style={font=\scriptsize},
]

% \nextgroupplot[ylabel={DVFS \\ Frequency \\ (GHz)}, % Performance
% xtick={0,500,1000,1500,2000,2500,3000,3500,4000},
% ytick={1.2,1.3,1.4,1.5,1.6,1.7,1.8,1.9,2.0,2.1},
% yticklabels={,,1.4,,1.6,,1.8,,~~~2.0,},
% % yticklabel style={font=\footnotesize},
% ymin=1.2,
% ymax=2.101,
% % legend entries={{\footnotesize $\mathsf{Server}$}},
% % legend style={draw=none,at={(0.5,1.4)},anchor=north,legend columns=4,line width=5pt},
% ]
% \addplot[thick, solid, color=s2] table[x index=0,y index=1,col sep=tab] {img/hipmer-ee.txt};
% \addplot[thick, solid, black] coordinates {(1.0, 2.1) (3640, 2.1)};
% % \addplot[thick, dashed, black] coordinates {(1500,0) (1500, 2)};
% % \addplot[thick, dashed, black] coordinates {(3000,0) (3000, 2)};


\nextgroupplot[ylabel={Energy \\ Efficiency \\ (Inst / J) }, % Power
ytick={0,250,500,750,1000,1250},
yticklabels={0,250,500,750,1000,1250},
% yticklabel style={font=\footnotesize},
ymin=0,
ymax=1250,
xlabel={$time$ [seconds]},
xlabel near ticks,
xtick={0,500,1000,1500,2000,2500,3000,3500,4000},
xticklabels={0,500,1000,1500,2000,2500,3000,3500,4000},
% xticklabel style={font=\footnotesize},
]
\addplot[thick, solid, color=s1] table[x index=0,y index=1,col sep=tab] {img/hipmer-ee.txt};
\addplot[thick, solid, black] coordinates {(0, 384.2304429183) (10000, 384.23)}; % race
% \addplot[thick, solid, black] coordinates {(0, 0.773) (60, 0.773)}; % worst-case
% \addplot[thick, dashed, black] coordinates {(1500,0) (1500, 250)};
% \addplot[thick, dashed, black] coordinates {(3000,0) (3000, 250)};

\end{groupplot}
\end{centering}

\end{tikzpicture}

  \caption{HiPMeraculous genome assembly application (Gradient Boosting classifier).}
  \label{fig:classifier-hipmer-ee}
  \vskip -1.8em
\end{figure}

\figref{classifier-hipmer-ee} demonstrates running a Gradient Boosting classifier which uses only 60\% power and saves 20\% energy over the naive race-to-idle approach, with a 33\% increase in runtime.
% The horizontal bar at 384 Inst/J represents the average energy efficiency of race-to-idle.
SGD and SVM performed similarly for this application, while KNN, RF, and SVM Linear achieve 55\% power, 10\% energy savings, and 60\% increase in runtime.
% These values equate to using only 50-60\% power at any given moment compared to race-to-idle.

Now consider hardware over-provisioned environments subject to power constraints, \eg DOE's 20 MW goal for exascale \cite{Exascale20MW}. %\cite{Exascale20MW, EOSR2012}.
Extrapolating from these results, we could perform nearly \emph{twice as much science} in parallel by better utilizing over-provisioned resources.
By trading performance for power savings, maximizing energy efficiency \emph{increases the total throughput of a hardware over-provisioned system by up to 50\%.}

\begin{tikzpicture}
\begin{centering}

\definecolor{s1}{RGB}{228, 26, 28}
\definecolor{s2}{RGB}{55, 126, 184}
\definecolor{s3}{RGB}{77, 175, 74}
\definecolor{s4}{RGB}{152, 78, 163}
\definecolor{s5}{RGB}{255, 127, 0}

\begin{groupplot}[
    group style={
        group name=plots,
        group size=1 by 2,
        xlabels at=edge bottom,
        xticklabels at=edge bottom,
        vertical sep=10pt
    },
height=8cm,
width=0.8\columnwidth,
xmajorgrids,
ymajorgrids,
grid style={dashed},
xmin=0,
xmax=10,
yticklabel pos=left,
enlargelimits=false,
tick align = outside,
tick style={white},
xticklabel shift={-5pt},
yticklabel shift={-5pt},
ylabel shift={-2pt},
ylabel style={align=center},
unbounded coords=jump,
legend cell align=left, 
legend style={ column sep=1ex },
]


\nextgroupplot[ylabel={Idle Time (S)}, % Performance
% xlabel={$time$ [iteration]},
% xlabel near ticks,
xtick={0,1,2,3,4,5,6,7,8,9,10},
ytick={0,2,4,6,8,10,12,14},
yticklabels={0,2,4,6,8,10,12,~~~14},
% yticklabel style={font=\footnotesize},
ymin=0,
ymax=14,
legend entries={{$\mathsf{Process~0}$},{$\mathsf{Process~1}$},{$\mathsf{Process~2}$},{$\mathsf{Process~3}$}},
legend style={draw=none,at={(0.5,1.25)},anchor=north,legend columns=4,line width=5pt},
]
\addplot[ultra thick, solid, color=s1] table[x index=0,y index=10,col sep=space] {img/nlpb.log};
\addplot[ultra thick, solid, color=s2] table[x index=0,y index=13,col sep=space] {img/nlpb.log};
\addplot[ultra thick, solid, color=s3] table[x index=0,y index=16,col sep=space] {img/nlpb.log};
\addplot[ultra thick, solid, color=s4] table[x index=0,y index=19,col sep=space] {img/nlpb.log};
\addplot[thick, solid, black] coordinates {(0, 45) (10, 45)};
% \addplot[ultra thick, dashed, black] coordinates {(1500,0) (1500, 2)};
% \addplot[ultra thick, dashed, black] coordinates {(3000,0) (3000, 2)};

\nextgroupplot[ylabel={Power Cap (W)}, % Performance
xlabel={$time$ [iteration]},
xlabel near ticks,
xtick={0,1,2,3,4,5,6,7,8,9,10},
ytick={44,44.25,44.5,44.75,45,45.25,45.5,45.75,46,46.25,46.5},
yticklabels={44.0,,44.5,,45.0,,45.5,,46.0,,},
% yticklabel style={font=\footnotesize},
ymin=44,
ymax=46.5,
% legend entries={{$\mathsf{Process~0}$},{$\mathsf{Process~1}$},{$\mathsf{Process~2}$},{$\mathsf{Process~3}$}},
% legend style={draw=none,at={(0.5,1.2)},anchor=north,legend columns=4,line width=5pt},
]
\addplot[ultra thick, solid, color=s1] table[x index=0,y index=12,col sep=space] {img/nlpb.log};
\addplot[ultra thick, solid, color=s2] table[x index=0,y index=15,col sep=space] {img/nlpb.log};
\addplot[ultra thick, solid, color=s3] table[x index=0,y index=18,col sep=space] {img/nlpb.log};
\addplot[ultra thick, solid, color=s4] table[x index=0,y index=21,col sep=space] {img/nlpb.log};
\addplot[thick, solid, black] coordinates {(0, 45) (10, 45)};
% \addplot[ultra thick, dashed, black] coordinates {(1500,0) (1500, 2)};
% \addplot[ultra thick, dashed, black] coordinates {(3000,0) (3000, 2)};


\end{groupplot}
\end{centering}

\end{tikzpicture}

\section{Conclusion}

Power and energy management is becoming more critical in HPC as we move toward exascale systems.
We proposed two projects to address balancing performance and power during runtime.
With over-provisioning, maximizing energy efficiency instead of using a race-to-idle approach allows completing more science in parallel by utilizing extra hardware that is constrained by a global power limit, like DOE's 20 MW goal for exascale.
Initial results indicate that throughput can be increased by up to 50\%.
We demonstrated that machine learning classifiers, driven by low-level hardware counters, can be used to predict energy-efficient settings at runtime to reduce both dynamic power and energy consumption.
We also proposed a MIMO PID controller to dynamically re-balance power allocations in a cluster to reduce tail idle times of parallel applications caused by non-uniformity in hardware and small workload imbalances.
Simulations indicate that this approach successfully reduces tail idle times and increases total application performance.


\paragraph*{Acknowledgements}
The University of
  Chicago's effort on this project is funded by the U.S. Government
  under the DARPA BRASS program, by the Dept. of Energy under DOE
  DE-AC02-06CH11357, by the NSF under CCF 1439156, and by a DOE Early
  Career Award.

% \balance
% \clearpage

% \bibliographystyle{ACM-Reference-Format}
% \bibliography{../seec}
\printbibliography

\end{document}
