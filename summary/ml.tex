% \begin{figure}[t]
%   \includegraphics[width=0.8\columnwidth]{../figures/seec.png}
%   \caption{The Self-aware Computing Model.}
%   \label{fig:runtime-ee}
% \end{figure}

\section{Machine Learning Classification for Energy Efficiency}

Tuning \textbf{core allocation (taskset)} and \textbf{dynamic voltage and frequency scaling (DVFS)} can save power and energy by power gating unneeded cores and scaling back active processors, \eg during I/O or for code not on the critical path.
Both are available in HPC job schedulers, though unfortunately today they are typically only configurable when a job is launched.

Other works have successfully used performance counters to manage performance and power consumption \cite{WuHPCComputer,Chetsa,Libutti2014}, including some that use statistical and machine learning approaches \cite{Sasaki,ShyeIDVFS,Alvarado,Curtis-Maury2008,LiIPDPS2010}.
However, these works mainly use estimation techniques that require significant modeling and computation.
By treating the goal of maximizing energy efficiency as a classification problem, we can avoid this pitfall.
Additionally, maximizing energy efficiency provides an optimal balance of performance and energy consumption while reducing dynamic power so it can be reallocated elsewhere.

\begin{figure}[t]
  \tikzset{%
  app/.style    = {draw, thin, rectangle, minimum height = 2em,
    minimum width = 2em, fill=black!25},
  block/.style    = {draw, thick, rectangle, minimum height = 2.5em,
    minimum width = 2.5em},
  blockres/.style    = {draw, thick, rectangle, minimum height = 2.5em,
    minimum width = 2.5em, fill=green!25},
  biblock/.style  = {draw, thick, rectangle, minimum height = 5.5em,
    minimum width = 6em, fill=red!25},
  sum/.style      = {draw, circle, node distance = 2cm}, % Adder
  input/.style    = {coordinate}, % Input
  output/.style   = {coordinate} % Output
}

\begin{tikzpicture}[scale=1.0,transform shape, auto, thick, node distance=1.5cm, >=triangle 45]

\draw
  % Drawing the top blocks
  % node [input, name=goalaccuracy] {} 
  % node [left of=goalaccuracy, node distance=0.35mm]{}
  % node [sum, right of=goalaccuracy] (sumaccuracy) {} % negative feedback
  node [block, align=center] (featureselection) 
    {Feature\\Selection}
  node [block, right of=featureselection, align=center, node distance=3cm] (classifier) 
    {Classifier}
  node [blockres, above of=classifier, align=center, node distance=1.5cm] (trainingdata) 
    {Training\\Data}
;
  % Connectng lines
% \draw[->](goalaccuracy) -- node[align=center] {Timing\\Goal}(sumaccuracy);
% \draw[->](sumaccuracy) -- node[align=center] {Timing\\Error}(featureselection);
\draw[->](featureselection) -- node[align=center] {Processed\\Data}(classifier);
\draw[->](trainingdata) -- (classifier);

% Draw software system
\draw
  node [biblock, right of=classifier, node distance=3cm, align=center, yshift=-0.2cm] (system)
    {\\System\\\\\\}
;
\draw
  node [app, right of=classifier, node distance=3cm, align=center, yshift=-0.5cm] (software)
    {Application}
;

% lines from translators to software
\draw[->](classifier.east) -- node [name=ka,align=center]{System\\Settings} (classifier.east -| system.west);

% Connectng lines
\coordinate (feedbackup) at ([yshift=-0.6cm]featureselection.south);
\draw (system.west |- feedbackup) -| node [near end,align=center] {PCM\\Sample} (feedbackup);
\draw[->](feedbackup) -- node[pos=0.99] {} (featureselection);

\end{tikzpicture}
  \caption{Machine Learning Classifier feedback control design.}
  \label{fig:runtime-ee}
\end{figure}

We propose using \interface{Machine Learning classifiers} to predict the most energy-efficient combination of taskset and DVFS settings \emph{during runtime} to maximize the work-to-energy ratio of an application running on a single node.
\figref{runtime-ee} demonstrates the feedback design.
Low-level \emph{hardware counter metrics}, available through tools like \app{PAPI} and \app{PCM}, train and drive our classifiers.
At regular time intervals, a classifier predicts the most energy-efficient setting to use based on measured application behavior, then applies the setting to the system.
As applications move through \emph{phases}, the predictions change accordingly.

We evaluate our approach on a quad-socket (160-logical-core) system with Intel Xeon E7-8870 processors and 512 GB of DRAM using HPC bioinformatic applications, which are often run on such large systems.
Naturally, it is important to demonstrate results on a single node before scaling to multiple nodes.
Training data is collected from characterizations of smaller HPC applications like the NAS Parallel Benchmarks, CoMD, HPGMG-FV, LULESH, and STREAM.
Principal Component Analysis determines which features are most relevant to energy efficiency---DRAM power, instructions per nominal CPU cycle (EXEC), L2/L3 cache hit/miss rates, and relative frequency excluding sleep time (AFREQ) are among the most useful.
We evaluate 6 common classifiers -- Gradient Boosting (GB), K-Nearest Neighbors (KNN), Random Forest (RF), Stochastic Gradient Descent (SGD), Support Vector Machine (SVM), and a Linear SVM.

\begin{figure}[t]
  \begin{tikzpicture}
\begin{centering}

\definecolor{s1}{RGB}{228, 26, 28}
\definecolor{s2}{RGB}{55, 126, 184}
\definecolor{s3}{RGB}{77, 175, 74}
\definecolor{s4}{RGB}{152, 78, 163}
\definecolor{s5}{RGB}{255, 127, 0}

\begin{groupplot}[
    group style={
        group name=plots,
        group size=1 by 1,
        xlabels at=edge bottom,
        xticklabels at=edge bottom,
        vertical sep=10pt
    },
height=8cm,
width=0.8\columnwidth,
xmajorgrids,
ymajorgrids,
grid style={dashed},
xmin=0,
xmax=3640,
yticklabel pos=left,
enlargelimits=false,
tick align = outside,
tick style={white},
xticklabel shift={-5pt},
yticklabel shift={-5pt},
ylabel shift={-2pt},
ylabel style={align=center},
unbounded coords=jump,
]

% \nextgroupplot[ylabel={DVFS \\ Frequency \\ (GHz)}, % Performance
% xtick={0,500,1000,1500,2000,2500,3000,3500,4000},
% ytick={1.2,1.3,1.4,1.5,1.6,1.7,1.8,1.9,2.0,2.1},
% yticklabels={,,1.4,,1.6,,1.8,,~~~2.0,},
% % yticklabel style={font=\footnotesize},
% ymin=1.2,
% ymax=2.101,
% % legend entries={{\footnotesize $\mathsf{Server}$}},
% % legend style={draw=none,at={(0.5,1.4)},anchor=north,legend columns=4,line width=5pt},
% ]
% \addplot[ultra thick, solid, color=s2] table[x index=0,y index=1,col sep=tab] {summary/img/hipmer-ee.txt};
% \addplot[thick, solid, black] coordinates {(1.0, 2.1) (3640, 2.1)};
% % \addplot[thick, dashed, black] coordinates {(1500,0) (1500, 2)};
% % \addplot[thick, dashed, black] coordinates {(3000,0) (3000, 2)};


\nextgroupplot[ylabel={Energy \\ Efficiency \\ (Inst / J) }, % Power
ytick={0,250,500,750,1000,1250},
yticklabels={0,250,500,750,1000,1250},
% yticklabel style={font=\footnotesize},
ymin=0,
ymax=1250,
xlabel={$time$ [seconds]},
xlabel near ticks,
xtick={0,500,1000,1500,2000,2500,3000,3500,4000},
xticklabels={0,500,1000,1500,2000,2500,3000,3500,4000},
% xticklabel style={font=\footnotesize},
]
\addplot[thick, solid, color=s1] table[x index=0,y index=1,col sep=tab] {summary/img/hipmer-ee.txt};
\addplot[thick, solid, black] coordinates {(0, 384.2304429183) (10000, 384.23)}; % race
% \addplot[thick, solid, black] coordinates {(0, 0.773) (60, 0.773)}; % worst-case
% \addplot[thick, dashed, black] coordinates {(1500,0) (1500, 250)};
% \addplot[thick, dashed, black] coordinates {(3000,0) (3000, 250)};

\end{groupplot}
\end{centering}

\end{tikzpicture}

  \caption{HiPMeraculous genome assembly application (Gradient Boosting classifier).}
  \label{fig:classifier-hipmer-ee}
  \vskip -1.8em
\end{figure}

\figref{classifier-hipmer-ee} demonstrates running a Gradient Boosting classifier which uses only 60\% power and saves 20\% energy over the naive race-to-idle approach, with a 33\% increase in runtime.
% The horizontal bar at 384 Inst/J represents the average energy efficiency of race-to-idle.
SGD and SVM performed similarly for this application, while KNN, RF, and SVM Linear achieve 55\% power, 10\% energy savings, and 60\% increase in runtime.
% These values equate to using only 50-60\% power at any given moment compared to race-to-idle.

Now consider hardware over-provisioned environments subject to power constraints, \eg DOE's 20 MW goal for exascale \cite{Exascale20MW}. %\cite{Exascale20MW, EOSR2012}.
Extrapolating from these results, we could perform nearly \emph{twice as much science} in parallel by better utilizing over-provisioned resources.
By trading performance for power savings, maximizing energy efficiency \emph{increases the total throughput of a hardware over-provisioned system by up to 50\%.}
